\documentclass[a4paper,twoside,titlepage]{article}
\usepackage[hmarginratio=1:1]{geometry}
\setcounter{secnumdepth}{3}
\usepackage{graphicx}
\usepackage{titlepic}
\usepackage[dvipsnames]{xcolor}
\usepackage{listings}
\usepackage{tikz}
\usetikzlibrary{positioning,shapes,shadows,arrows,fit,decorations.markings}
\usepackage{fancyhdr}
\pagestyle{fancy}
\usepackage[
	pdftitle={Muen},
	pdfsubject={Toolchain},
	pdfauthor={Reto Buerki, Adrian-Ken Rueegsegger},
	pdfkeywords={Separation Kernel, Toolchain, Build, System Image, XML},
	unicode=true,
	bookmarks=true,
	bookmarksnumbered=false,
	bookmarksopen=false,
	breaklinks=true,
	pdfborder={0 0 0},
	backref=false,
	colorlinks=false]{hyperref}

\lstset{
	basicstyle={\ttfamily\scriptsize},
	breakautoindent=true,
	breaklines=true,
	captionpos=b,
	extendedchars=true,
	frame=single,
	numbers=left,
	numberstyle={\tiny},
	showspaces=false,
	showstringspaces=false,
	tabsize=2,
	keywordstyle={\color{MidnightBlue}},
	commentstyle={\color{Aquamarine}},
	literate={~} {$\sim$}{1}
}

\title{Muen - Toolchain}
\author{Reto Buerki \and Adrian-Ken Rueegsegger}
\titlepic{\includegraphics[scale=0.4]{../images/muen.pdf}}

\begin{document}
\input{tikzstyle}

\maketitle

\begin{center}
	University of Applied Sciences Rapperswil (HSR), Switzerland
\end{center}
\clearpage

\tableofcontents
\listoffigures

\section{Introduction}
This document describes the process of configuring and building a
component-based system running on the Muen Separation Kernel (SK).

\section{Policy}
A policy is a description of a component-based system running on top of the Muen
Separation Kernel. It defines what hardware resources are present, how many
active components (called subjects) the system is composed of, how they interact
and which system resources they are allowed to access. The following properties
are specified by the policy:

\begin{itemize}
	\item Configuration values
	\item Hardware resources
	\item Platform description
	\item Kernel diagnostics device
	\item Physical memory regions
	\item Device domains
	\item Events
	\item Communication channels
	\item Components
	\item Subjects
	\item Scheduling plans
\end{itemize}

The policy serves as a static description of a Muen system. Since all aspects of
the system are fixed at integration time the policy can be validated prior to
execution, see also section \ref{subsec:policy_validation}.

\subsection{Content}
This section presents the different parts of a system policy and gives an
overview of what each section contains.

\subsubsection{Configuration Values}
The purpose of a config section is to specify configuration values which
parameterize a system or a component. It allows to declare boolean, string and
integer values, e.g. \texttt{<boolean name="iommu\_enabled" value="true"/>}.

The following sections provide support for configuration values:
\begin{itemize}
	\item System
	\item Platform
	\item Component
\end{itemize}

During the build process, configuration values provided by the platform are
merged into the global system configuration. Component configuration values
allow the parameterization of component-local functionality.

\subsubsection{Hardware Resources}
Systems running the Muen SK perform static resource allocation at integration
time. This means that all available hardware resources of a target machine must
be defined in the system policy in order for these resources to be allocated to
subjects.

Data required by a hardware description includes the amount of available
physical memory blocks including reserved memory regions (RMRR), the number of
logical CPUs and hardware device resources.

The Muen toolchain provides a handy tool to automate the cumbersome process of
gathering hardware resource data, see section \ref{sec:mugenhwcfg}.

\subsubsection{Platform description}
To enable a uniform view of the hardware resources across different physical
machines from the system integrators perspective, the platform description layer
is interposed between the hardware resource description and the rest of the
system policy. This allows to build a Muen system for different physical target
machines using the same system policy.

The config section enables the declaration of platform-specific properties.

\subsubsection{Kernel diagnostics device}
The Muen SK can be instructed to output debugging information during runtime.
The kernel diagnostics device specifies which I/O device the kernel is to use
for this purpose.

\subsubsection{Physical Memory Regions}
This part of the policy specifies the physical memory layout of the system.
Memory regions are defined by their size, caching, type and are placed by
specifying a physical address. Additionally the content of the region can be
declared as backed by a file or filled with a pattern.

\subsubsection{Device Domains}
The physical memory accessible by PCI devices is specified by so called device
domains. Such domains define virtual mappings of physical memory regions for one
or multiple devices. Device references select a subset of hardware devices
provided by the platform.

Device domains are isolated from each other by the use of Intel VT-d. Thus they
can only be specified and enforced on systems that provide at least one
IOMMU\footnote{Input/Output Memory Management Unit}.

\subsubsection{Events}
Events are an activity caused by a subject (source) that impacts a second
subject (target) or is directed at the kernel. Events are declared globally and
have a unique name to be unambiguous.

Subjects can use events to either deliver an interrupt, hand over execution to
or reset the state of a target subject. The first kind of event provides a
basic notification mechanism and enables the implementation of event-driven
services. The second type facilitates suspension of execution of the source
subject and switching to the target. Such a construct is used to pass the
thread of execution on to a different subject, e.g. invocation of a debugger
subject if an error occurs in the source subject. The third kind is used to
facilitate the restart of subjects.

Kernel events are special in that they are targeted at the kernel. The
currently supported events are system reboot and shutdown.

\subsubsection{Communication Channels}
Inter-subject communication is represented by so called channels. These channels
represent directed information flows since they have a single writer and
possibly multiple readers. Optionally a channel can have an associated
notification event (doorbell interrupt).

Channels are declared globally and have a unique name to be unambiguous.

\subsubsection{Components}
A component is a piece of software executed by the SK. Similar terms are
partition or container. They represent the building blocks of a component-based
system.

The description of a component specifies the binary program file including the
virtual memory location as well as the view of the expected execution
environment. This environment is defined in terms of logical resources such as
for example communication channels.

\subsubsection{Subjects}
Subjects are instances of components. A subject specification references a
component and maps the declared logical resources to physical resources provided
by the system.

Besides the component resource mappings, subjects can specify extra resources
such as device and/or memory mappings. This is useful for subjects which are
able to enumerate the available resources at runtime via configuration
mechanisms like ACPI or the Subject Information Page.

Furthermore, subject specifications enable the declaration of events a subject
is allowed to trigger and receive.

Subjects also have an associated profile (e.g. native or Linux) which determines
properties of the execution environment provided by the kernel.

\subsubsection{Scheduling Plans}
The Muen SK performs scheduling of subjects in a fixed, cyclic and
preemptive way according to a user-specified regime. Scheduling information is
declared in so called scheduling plans. They specify in what order subjects are
executed on which logical CPU and for how long. Multiple scheduling plans can be
specified to enable the definition of different system execution profiles which
can be switched during runtime.

A scheduling plan is specified in terms of frames. A \emph{major frame} consists
of a sequence of minor frames. When the end of a major frame is reached, the
scheduler starts over from the beginning and uses the first minor frame in a
cyclic fashion. This means that major frames are repetitive. A
\emph{minor frame} specifies a subject and a precise amount of time.

The \texttt{mugenschedcfg} tool can be used to automatically generate
scheduling plans from a given scheduling configuration, see section
\ref{sec:mugenschedcfg}.

\subsection{Format}
The system policy is specified in XML. There are currently three different
policy formats:

\begin{itemize}
	\item Source Format
	\item Format A
	\item Format B
\end{itemize}

The motivation to have several policy formats is to provide abstractions and a
compact way for users to specify a system while simultaneously facilitating
reduced complexity of tools operating on the policy.

The implementation of such tools is simplified by the absence of higher-level
abstractions which would make the extraction of input data more involved. As
an example, the page table generation tool can directly access all virtual
memory mappings of a subject and must not concern itself with channels. The
channel abstraction has already been broken down into the corresponding memory
elements during the policy compilation step of the build process (see
section \ref{sec:build-policy_compilation}).

\subsubsection{Source Format}
The user-specified policy is written in the source format. Constructs such as
channels or events provide abstractions to simplify the specification of
component-based systems. Many XML elements and attributes are optional and will
be filled in with default values during later steps of the policy compilation
process.

Kernel and $\tau$0 resources are not part of the source format since they are also
automatically added by the policy expansion step.

Additionally, the use of configuration values enables easy parametrization of
the system policy.

\subsubsection{Format A}
Format A is a processed version of the source format where all includes are
resolved and abstractions such as channels have been broken down into their
underlying elements. For example, a channel is expanded to a physical memory
region and the corresponding writer and reader subject mappings with the
appropriate access rights.

In this format all implicit elements, such as for example automatically
generated page table memory regions, are specified. The kernel and $\tau$0
configuration is also declared as part of format A.

The only optional attributes are addresses of physical memory regions.

\subsubsection{Format B}
Format B is equivalent to Format A except that all physical memory regions have
a fixed location (i.e. their physical address is set).

\section{Build Process}
The build of a system is divided into the following steps:

\begin{itemize}
	\item Policy merging
	\item Components build
	\item Components spec merging
	\item Policy compilation
	\item Policy validation
	\item Structure generation
	\item Image packaging
\end{itemize}

The toolchain is composed of several tools that operate on a user-specified
system policy. Following the Unix philosophy "A program should do only one thing
and do it well" each of the tools performs a specific task. They work in
conjunction to process a user-defined policy and build a bootable system image.
An in-depth description of the involved tools is given in section
\ref{sec:tools} while figure \ref{fig:build-process} gives an overview of the
whole build process.

\begin{figure}[h!]
	\centering
	\input{graph_build_process}
	\caption{Build process}
	\label{fig:build-process}
\end{figure}

\subsection{Policy Merging}
\label{sec:build-policy_merging}
The Merger tool outlined in section \ref{sec:tools-merger} is responsible to
merge XML files stored at different locations on the file system into one
system policy in source format.

The tool reads a system configuration in XML format to locate the following
files:
\begin{itemize}
	\item System policy
	\item Hardware specification
	\item Additional hardware specification
	\item Platform specification
\end{itemize}

The tool also provides an implementation of the XML XInclude
mechanism\footnote{\url{http://www.w3.org/TR/xinclude-11/}}. Using includes,
the policy writer is able to separate and organize the system policy as
desired. Instead of specifying the whole policy in one file, the subject
specifications can be split into separate files, or common parts shared by
different system descriptions can be extracted. See section
\ref{sec:tools-merger} for more information about the Merger tool.

After the merge step, the resulting policy is well formatted to minimize the
difference in the generated policies resulting from the subsequent tasks. This
allows the user to easily review (\texttt{diff}) and therefore verify the
results of each policy compilation task.

Expressions can be used to formulate (nested) boolean terms using the numeric
equality/inequality and logical operators. They are evaluated to boolean config
values prior to processing conditionals.

The use of conditionals enables selective activation of parts of the source
policy depending on the value of a given config variable. This allows flexible
customization of a system during policy compilation time by setting the value
of a config variable or formulating an appropriate boolean expression.

Config variable substitution enables the policy writer to set the value of
attributes to those of referenced config values. Attributes that start with a
dollar sign followed by a config value name are substituted by the value of the
config variable.

\subsection{Components Build}
\label{sec:build-components}
After hardware, platform and high-level system policy are merged, components
can extract relevant information from it using whatever means suit best. For
example an XSL transformation (XSLT) script could extract the I/O port of a
specific device and create a corresponding configuration value based on it,
which is then included in the component specification.

The \texttt{mucgenspec} tool described in \ref{sec:tools-cgen-spec} implements
the blue \emph{Specgen} task shown in figure \ref{fig:build-process}. It is
used to process component specifications and, similar to the policy merger,
supports conditionals, expressions and configuration value substitutions. It
also generates Ada/SPARK packages containing constants derived from the
declared component resources and config values. These constants can be used to
reliably address specific or configurable resources in the source code.

After the component specification has been processed, the component source code
is compiled into a binary.

The \texttt{mucbinsplit} tool described in \ref{sec:mucbinsplit} can be used to
extract ELF sections of the component binary into separate files. It
automatically extends the component specification by adding a corresponding
memory region with the appropriate access rights (e.g. executable, writable).

\subsection{Components Specification Merging}
\label{sec:build-compspec-merging}
The processed component specifications are merged into the system source
policy by the Muen component specification joiner tool described in section
\ref{sec:tools-comps-joiner}.

This step is optional as static component specifications which need no
processing can also be manually specified in the system policy directly.

\subsection{Policy Compilation}
\label{sec:build-policy_compilation}
Policy compilation encompasses the tasks involved to transform the policy from
source format to format A and finally to format B, which is the fully expanded
format with no implicit properties.

The Expander tool takes care of completing the user-specified policy with
additional information and abstractions only available in format source are
resolved to low-level constructs.

For example, the concept of \emph{channels} only exists in format source.
Therefore a channel specified in format source must be expanded to shared memory
regions with optional associated events in format A.  Also, the Expander tool
inserts specifications for the Muen kernel itself so the user is lifted from
that burden. Generally, the aim of the expansion task is to make the life of a
policy writer as easy as possible by expanding all information which can be
derived automatically. Section \ref{sec:tools-expander} explains the Expander
tool in detail.

The result of the expansion task is a policy in format A which is the input for
the Allocator tool. This tool is responsible to assign a physical memory
address to all memory regions which are not already explicitly stated. By
querying the hardware section of the policy, the tool is aware of the total
amount of available RAM on a specific system and allocates regions of it for
memory elements with no explicit physical address.  The Allocator tool also
implements optimization strategies to keep the resulting system image as small
as possible.  For example, file-backed memory regions (e.g. a memory region
storing a component executable) are preferably placed in lower physical
regions. See section \ref{sec:tools-allocator} for a description of the
Allocator tool.

After the allocation task is complete, the policy is stored in format B. This
format states all system properties explicitly and is used as input for the
Validation step discussed in the following section.

\subsection{Policy Validation}
\label{subsec:policy_validation}
Before structures required to pack the final system image are generated, the
policy must be thoroughly validated to catch errors in the system
specification.  Such errors might range from overlapping memory, undefined
resource references to incomplete scheduling plans etc. The Validator task
performs checks that assure the policy in format B is sound and free from
higher-level errors that are not covered by XML schemata restrictions.

It is important to always run the Validator as the system could otherwise
exhibit unexpected behavior. This is especially true if a policy writer decides
to specify the system directly in format B which is also possible but not
advised. Section \ref{sec:tools-validator} explains the Validator tool and
lists some example checks performed by the tool as illustration.

\subsection{Structure Generation}
The structure generation step encompasses various tools which extract
information from a policy in format B and generate files in different formats
(see figure \ref{fig:build-process}).

While some generated files are directly linked into the Muen kernel (i.e.
Source Specs, see \ref{sec:tools-gen-spec}), most of them are packed into the
final system image by the packer tool.

For example, the tool responsible to generate page table structures queries
memory mappings and the associated physical memory regions from the policy and
creates page table structures according to the format specified by the Intel
Software Developer's Manual (SDM). The resulting files are packed into the
system image and only applied by the kernel. The kernel itself does not care
about memory management, all required tables are pre-built during system
integration.

For more information about the structure generators, see section
\ref{sec:tools-gen}.

\subsection{Image Packaging}
The Packer tool assembles the final system image by first allocating a memory
buffer which is initialized to zero. The size of the buffer is large enough to
hold the complete system image, which consists of all file-backed memory
regions specified in the policy:

\begin{itemize}
	\item Kernel binary
	\item Kernel page tables
	\item I/O bitmaps
	\item MSR bitmaps
	\item MSR store
	\item Subject binaries
	\item Subject page tables
	\item VT-d tables
	\item ACPI tables for VM subjects
	\item Initial Ramdisks for Linux subjects
	\item Zero Page structures for Linux subjects
	\item Solo5 boot info structures for MirageOS subjects
	\item Muen subject information structures
\end{itemize}

It then simply iterates over all file-backed memory regions and inserts the
contents of the specified files into the allocated buffer. After performing
various post-checks on the created image, it is written to a file. The
resulting image can be booted by any
Multiboot\footnote{\url{https://www.gnu.org/software/grub/manual/multiboot/multiboot.html}}
compliant bootloader.

For more information about the Packer tool, see section \ref{sec:tools-packer}.

\section{Core Tools}
\label{sec:tools}

This section describes the tools which form the core of the Muen toolchain.

\subsection{Merger}
\label{sec:tools-merger}
The Merger combines user-provided system policy files into a single XML
document.

\begin{description} \itemsep1pt \parskip0pt
	\item[Name] \hfill \\
		\texttt{mucfgmerge}
	\item[Input] \hfill \\
		System configuration as XML, Colon-separated list of include paths
	\item[Output] \hfill \\
		System policy in format source (merged)
\end{description}

This tool reads the system configuration and merges the specified system
policy, hardware and platform files into a single file. It evaluates boolean
expressions, resolves conditional parts of the policy and substitutes attribute
configuration value references. Included files are inserted at the
corresponding locations in the merged file. The XML content is re-formatted so
changes to the policy by subsequent build steps can be manually reviewed or
visualized by diffing the files.

It also merges the platform configuration section (if any) into the global
configuration section, removing the platform configuration section in the
process.

\subsection{Component Joiner}
\label{sec:tools-comps-joiner}
The Muen component specification joiner adds component XML specifications to
the component section of a specified system policy and writes the result to a
specified output file. Each given component/library specification is loaded and
validated against the components XML schema. If it is correct the content is
added to the components section of the system policy specified as input file.
If the given system policy does not yet contain a components section, it is
created. The result is written to the file specified by the \texttt{-o}
parameter.  In-place processing is supported by passing in the same value for
input and output file.

\begin{description} \itemsep1pt \parskip0pt
	\item[Name] \hfill \\
		\texttt{mucfgcjoin}
	\item[Input] \hfill \\
		System policy in format source, comma-separated list of component specs
	\item[Output] \hfill \\
		System policy in format source (joined)
\end{description}

\subsection{Expander}
\label{sec:tools-expander}
The expander completes the user-provided system policy by creating or deriving
additional configuration elements.

\begin{description} \itemsep1pt \parskip0pt
	\item[Name] \hfill \\
		\texttt{mucfgexpand}
	\item[Input] \hfill \\
		System policy in format source
	\item[Output] \hfill \\
		System policy in format A (expanded)
\end{description}

The Expander performs the following actions:
\begin{itemize}
	\item Pre-check the system policy to make sure it is sound
	\item Expand channels
	\item Expand device resources
	\item Expand device isolation domains
	\item Expand kernel sections
	\item Expand $\tau$0 subject
	\item Expand additional memory regions
	\item Expand hardware-/platform-related information
	\item Expand additional subject information
	\item Expand profile-specific information
	\item Expand scheduling information
	\item Post-check resulting policy
\end{itemize}

\subsection{Allocator}
\label{sec:tools-allocator}
The Allocator is responsible to assign a physical address to all global memory
regions.

\begin{description} \itemsep1pt \parskip0pt
	\item[Name] \hfill \\
		\texttt{mucfgalloc}
	\item[Input] \hfill \\
		System policy in format A
	\item[Output] \hfill \\
		System policy in format B (allocated)
\end{description}

First, the Allocator initializes the physical memory view of the system based
on the physical memory blocks specified in the XML hardware section. It then
reserves memory that is occupied by pre-allocated memory elements (i.e. memory
regions with a physical address or device memory).  Finally it places all
remaining memory regions in physical memory.  In order to reduce the size of
the final system image file-backed memory regions are placed at the start of
memory.

\subsection{Validator}
\label{sec:tools-validator}
The Validator performs additional checks that go beyond the basic restrictions
imposed by the XML schema validation. Currently over 110 checks are performed.

\begin{description} \itemsep1pt \parskip0pt
	\item[Name] \hfill \\
		\texttt{mucfgvalidate}
	\item[Input] \hfill \\
		System policy in format B
	\item[Output] \hfill \\
		None, raises exception on error
\end{description}

Examples of checks include:

\begin{itemize}
	\item Assert that references between policy elements are correct (e.g. a
		physical memory	region referenced by a virtual memory region exists)
	\item Assert that memory regions do not overlap
	\item Assert that device interrupts are unique
	\item Assert that no subject has access to system or kernel memory
	\item Assert that scheduling plan major frames have the same length on each
		CPU
\end{itemize}

\subsection{Structure Generators}
\label{sec:tools-gen}
These tools do not change the policy and use it read-only.

\subsubsection{Page Tables}
Generate page tables for kernel(s) and subjects.

\begin{description} \itemsep1pt \parskip0pt
	\item[Name] \hfill \\
		\texttt{mugenpt}
	\item[Input] \hfill \\
		System policy in format B
	\item[Output] \hfill \\
		Page tables of kernels and subjects in binary format
	\item[Output format] \hfill
		\begin{itemize}
			\item IA32-e paging structures, Intel SDM Vol. 3A, section 4.5
			\item EPT paging structures, Intel SDM Vol. 3C, section 28.2
		\end{itemize}
\end{description}

The tool generates paging structures for subjects and kernels running on each
CPU. These page tables are used to grant access to physical memory according to
the virtual memory layout of the subject. The rest of physical and device
memory is isolated from the subject.

An IA32-e page table is generated for each kernel running on a logical, active
CPU. Depending on the subject profile either native 64-bit IA32-e or Extended
Page Tables (EPT) are generated.

Page tables are used by the memory management unit (MMU) to enforce isolation
of physical memory according to the system policy.

\subsubsection{VT-d Tables}
Generate VT-d tables for each device isolation domain.

\begin{description} \itemsep1pt \parskip0pt
	\item[Name] \hfill \\
		\texttt{mugenvtd}
	\item[Input] \hfill \\
		System policy in format B
	\item[Output] \hfill \\
		VT-d tables of device domains in binary format
	\item[Output format] \hfill \\
		VT-d tables according to Intel VT-d specification, section 9
\end{description}

The tool creates root, context and second-level address translation tables for
Intel VT-d DMAR (DMA\footnote{DMA - Direct Memory Access} remapping) hardware
(see Intel VT-d specification, section 3). DMAR is used to restrict direct
hardware device access to physical memory via DMA. Devices are put in so-called
device security or device isolation domains and are only allowed to access
physical memory as granted by the policy.

Interrupt remapping tables are also generated for Intel VT-d IR to ensure that
physical devices can only generate interrupt requests as specified by the system
policy.

\subsubsection{I/O Bitmaps}
Generate I/O bitmaps for each subject.

\begin{description} \itemsep1pt \parskip0pt
	\item[Name] \hfill \\
		\texttt{mugeniobm}
	\item[Input] \hfill \\
		System policy in format B
	\item[Output] \hfill \\
		I/O bitmaps of subjects in binary format
	\item[Output format] \hfill \\
		Intel SDM Vol. 3C, section 24.6.4
\end{description}

The tool generates I/O bitmaps for each subject. Access to device I/O ports is
granted according to the device I/O port resources assigned to a subject.

I/O bitmaps are used by the hardware (VT-x) to enforce access to I/O ports
according to the system policy.

\subsubsection{MSR Bitmaps}
Generate MSR bitmap for each subject.

\begin{description} \itemsep1pt \parskip0pt
	\item[Name] \hfill \\
		\texttt{mugenmsrbm}
	\item[Input] \hfill \\
		System policy in format B
	\item[Output] \hfill \\
		MSR bitmaps of subjects in binary format
	\item[Output format] \hfill \\
		Intel SDM Vol. 3C, section 24.6.9
\end{description}

The tool generates MSR bitmaps for each subject. Access to Model-Specific
Registers (MSRs) is granted according to the MSRs assigned to a subject.

MSR bitmaps are used by the hardware (VT-x) to enforce access to Model-Specific
Registers according to the system policy.

\subsubsection{MSR Stores}
Generate MSR store for each subject with MSR access.

\begin{description} \itemsep1pt \parskip0pt
	\item[Name] \hfill \\
		\texttt{mugenmsrstore}
	\item[Input] \hfill \\
		System policy in format B
	\item[Output] \hfill \\
		MSR store files of subjects in binary format
	\item[Output format] \hfill \\
		Intel SDM Vol. 3C, table 24-11
\end{description}

The tool generates MSR stores for each subject. The MSR store is used to
save/load MSR values of registers not implicitly handled by hardware on subject
exit/resumption.

MSR stores are used by hardware (VT-x) to enforce isolation of MSR (i.e.
subjects that have access to the same MSRs cannot transfer data via these
registers).

\subsubsection{ACPI Tables}
Generate ACPI tables for all Linux subjects.

\begin{description} \itemsep1pt \parskip0pt
	\item[Name] \hfill \\
		\texttt{mugenacpi}
	\item[Input] \hfill \\
		System policy in format B
	\item[Output] \hfill \\
		ACPI tables of all Linux subjects
	\item[Output format] \hfill \\
		Advanced Configuration and Power Interface (ACPI)
		Specification\footnote{\url{http://www.acpi.info/DOWNLOADS/ACPIspec50.pdf}}
\end{description}

ACPI tables are used to announce available hardware to VM subjects. A set of
tables consists of an RSDP, XSDT, FADT and DSDT table. See the ACPI
specification for more information about a specific table.

\subsubsection{Linux Zero Pages}
Generate Zero Pages for all Linux subjects.

\begin{description} \itemsep1pt \parskip0pt
	\item[Name] \hfill \\
		\texttt{mugenzp}
	\item[Input] \hfill \\
		System policy in format B
	\item[Output] \hfill \\
		Zero pages of all Linux subjects
	\item[Output format] \hfill \\
		Linux Boot Protocol\footnote{\url{https://www.kernel.org/doc/Documentation/x86/boot.txt}} \\
		Zero Page\footnote{\url{https://www.kernel.org/doc/Documentation/x86/zero-page.txt}}
\end{description}

The so-called Zero Page (ZP) exports information required by the boot protocol
of the Linux kernel on the x86 architecture. The kernel uses the provided
information to retrieve settings about its running environment:
\begin{itemize}
	\item Type of bootloader
	\item Map of physical memory (e820 map)
	\item Address and size of initial ramdisk(s)
	\item Kernel command line parameters
\end{itemize}

\subsubsection{Solo5 boot info}
Generate Solo5 boot info structures for Mirage unikernels running on the
Solo5 platform with the unikernel monitor backend.

\begin{description} \itemsep1pt \parskip0pt
	\item[Name] \hfill \\
		\texttt{mugensolo5}
	\item[Input] \hfill \\
		System policy in format B
	\item[Output] \hfill \\
		Solo5 boot info for all MirageOS subjects
	\item[Output format] \hfill \\
		struct hvt\_boot\_info\footnote{\url{https://github.com/Solo5/solo5/blob/master/include/solo5/hvt_abi.h}}
\end{description}

The boot info structure exports information required by Solo5. The
unikernel uses the provided information to retrieve settings about its runtime
environment:
\begin{itemize}
	\item Memory size in bytes
	\item Address of end of unikernel
	\item Command line parameters
	\item TSC frequency
\end{itemize}

\subsubsection{Source Specifications}
\label{sec:tools-gen-spec}
Generate source specifications used by kernel and subjects.

\begin{description} \itemsep1pt \parskip0pt
	\item[Name] \hfill \\
		\texttt{mugenspec}
	\item[Input] \hfill \\
		System policy in format B
	\item[Output] \hfill \\
		Source specifications in SPARK, C and GPR format
\end{description}

Gathers data from the system policy to generate various source files in SPARK,
C and GNAT project file (GPR) format. Created output includes constant values
for memory addresses, device resources, scheduling plans, etc.

\subsubsection{Component Source Specifications}
\label{sec:tools-cgen-spec}
Process component description and generate source specifications from it. Write
processed description to specified output file.

\begin{description} \itemsep1pt \parskip0pt
	\item[Name] \hfill \\
		\texttt{mucgenspec}
	\item[Input] \hfill \\
		Component description in XML, colon-separated list of include paths
	\item[Output] \hfill \\
		Component source specifications in SPARK, processed component
		description in XML
\end{description}

The component spec generation tool processes the given component description by
evaluating XIncludes, boolean expressions and resolving conditional parts.
Furthermore, it performs substitutions of attributes with configuration values.

It also generates Ada/SPARK packages containing constants of the declared
logical component resources. The generated specifications can be used in the
component source code to access the declared resources.

The resulting processed component description is written to the given output
location.

\subsubsection{Subject Info}
Generate subject information data for each subject.

\begin{description} \itemsep1pt \parskip0pt
	\item[Name] \hfill \\
		\texttt{mugensinfo}
	\item[Input] \hfill \\
		System policy in format B
	\item[Output] \hfill \\
		Subject info data in binary format
	\item[Output format] \hfill \\
		As specified in \texttt{common/musinfo/musinfo.ads}
\end{description}

The Sinfo page is used to export subject information data extracted from the
system policy to VM subjects. Currently, information about available memory
regions, communication channels and assigned PCI devices is provided.

\subsection{Hasher}
The Mucfgmemhashes tool is used to add memory integrity hashes to a given
policy.

\begin{description} \itemsep1pt \parskip0pt
	\item[Name] \hfill \\
		\texttt{mucfgmemhashes}
	\item[Input] \hfill \\
		System policy in format B
	\item[Output] \hfill \\
		System policy in format B with memory integrity hashes
\end{description}

The Mucfgmemhashes tool appends a hash to all memory regions with fill and file
content. It must run after all files have been generated by the structure
generator tools.

The actual hash is generated using the SHA-256 algorithm and is intended to be
used to verify the integrity of memory regions during runtime.

Note that no hashes are generated for sinfo memory regions. Since the
hash information will be exported via sinfo, and the sinfo region is
itself part of the memory information of a subject, this hash would be
self-referential.

The tool also replaces all occurrences of \texttt{hashRef} elements. A hash
reference element instructs the tool to copy the hash element of the referenced
memory region after message digest generation.

From an abstract point of view, the \texttt{hashRef} element is a way to link
multiple memory regions by declaring that the hash of the content is the same.
The hash may serve as an indicator on how to reconstruct the (initial) content
of a memory region. This mechanism is heavily used by the subject loader (SL)
during subject init and reset operation. The subject loader expander remaps
writable memory regions of the loadee (the subject under loader control) to SL
and replaces the original regions with new ones containing a hash reference to
the associated physical memory region. This way SL is able to determine the
intended content of the target memory region by looking up the region in its
sinfo page by using the hash value as key.

\subsection{Packer}
\label{sec:tools-packer}
The Packer is responsible for assembling the final system image.

\begin{description} \itemsep1pt \parskip0pt
	\item[Name] \hfill \\
		\texttt{mupack}
	\item[Input] \hfill \\
		System policy in format B, Input directories, System image filename
	\item[Output] \hfill \\
		System image file
\end{description}

The Packer calculates the size of the resulting system image by querying the
file-backed memory region with the highest physical memory address. It
allocates a buffer of that size which is initially filled with zeros. It then
iterates over all file-backed memory regions in the policy and adds the content
of the files to the buffer. Before writing the buffer to a file specified on
the command line, the packer tool performs post-checks on the buffer to make
sure it is sound.

\section{Additional Tools}
\label{sec:addtools}

This section lists additional helper tools which simplify the process of
generating and validating a Muen system.

\subsection{Kernel ELF Checker}
\label{sec:mucheckelf}
The \texttt{Mucheckelf} tool enforces that the format of a given Muen kernel
ELF binary matches the kernel memory layout specified in a system policy.

Size, VMA (Virtual Memory Address) and permissions of binary ELF sections are
validated against kernel memory regions defined in the policy. The following
table lists the correspondence of ELF section names to logical kernel memory
region names.

\begin{table}[h]
	\centering
	\begin{tabular}{l|l}
		\textbf{ELF Section} & \textbf{Memory Name} \\
		\hline
		.text   & kernel\_text \\
		.data   & kernel\_data \\
		.rodata & kernel\_ro   \\
		.bss    & kernel\_bss  \\
		\hline
	\end{tabular}
\end{table}

\subsection{Stack Usage Checker}
\label{sec:mucheckstack}
The \texttt{Mucheckstack} tool statically calculates the worst-case stack usage
of a native Ada/SPARK component or the Muen kernel compiled with the
-fcallgraph-info
switch\footnote{\url{https://www.adacore.com/uploads/technical-papers/Stack\_Analysis.pdf}}.

The tool takes a GNAT project file and a stack limit in bytes as input.  All
control-flow information (.ci) files found in the object directory of the main
project and all of its dependencies are parsed. Once the control-flow graph is
constructed the maximum stack usage of each subprogram is calculated and
checked against the user-specified limit. The tool exits with a failure if a
stack usage exceeding the limit is detected.

Note that the tool is not applicable to arbitrary software projects
as it does not handle dynamic/unbounded stack usage and recursion. In the
context of the Muen project these cases can not occur since they are prohibited
by the following restriction pragmas:
\begin{itemize}
	\item No\_Recursion
	\item No\_Secondary\_Stack
	\item No\_Implicit\_Dynamic\_Code
\end{itemize}

Additionally, the \texttt{-Wstack-usage} compiler switch warns about potential
unbounded stack usage.

\subsection{Hardware Config Generator}
\label{sec:mugenhwcfg}
The
\texttt{Mugenhwcfg}\footnote{\url{https://git.codelabs.ch/?p=muen/mugenhwcfg.git}}
tool has been created to automate the process of gathering all necessary
hardware information. To collect data for a new target hardware all that is
required is to run the tool on a common Linux distribution. See the project
README for more information.

\subsection{Scheduling Plan Generator}
\label{sec:mugenschedcfg}
The
\texttt{Mugenschedcfg}\footnote{\url{https://git.codelabs.ch/?p=muen/mugenschedcfg.git}}
tool generates scheduling plans for Muen based on a given scheduling
configuration.  The configuration allows the user to specify the following
scheduling properties:

\begin{itemize}
	\item Number of CPU cores
	\item The tick rate of the CPUs
	\item Security constraints to meet
	\begin{itemize}
		\item Same CPU domains
		\item Simultaneous execution domains
	\end{itemize}
	\item Subject specifications
	\item Score functions
	\item Number of plans to generate
	\item Plans
	\begin{itemize}
		\item Weighting of plan importance
		\item Levels
		\item Subjects of a plan
		\item Chains with throughput metric
	\end{itemize}
\end{itemize}

Consult the project's README and example plans on how to use the tool.

\subsection{Component Binary Splitter}
\label{sec:mucbinsplit}
The \texttt{mucbinsplit} tool splits component binaries into multiple files per
ELF section.

\begin{description} \itemsep1pt \parskip0pt
	\item[Name] \hfill \\
		\texttt{mucbinsplit}
	\item[Input] \hfill \\
		Component description in XML, Component ELF binary
	\item[Output] \hfill \\
		Binary files corresponding to ELF sections, processed component
		description in XML
\end{description}

The component binary splitter tool processes component binaries and creates a
separate file for each ELF section. The component XML description is extended
by adding a file-backed memory region for each ELF section with the appropriate
size and access rights.

The resulting processed component description is written to the given output
location while the binary section files are written to the specified output
path.

\end{document}
